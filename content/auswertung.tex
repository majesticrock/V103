\section{Auswertung}
\label{sec:Auswertung}

\begin{table}[!htp]
  \centering
  \caption{Eigenschaften der Stäbe und die jeweils angehängten Massen.}
  \label{tab:staebe}
  \begin{tabular}{c S[table-format=3.0] S[table-format=1.2] S[table-format=3.1] S[table-format=4.1]}
    \toprule
    {Stab} & {$r$ / mm} & {$l$ / m} & {$m_\text{Stab}$ / g} & {$m_\text{Gewicht}$ / g} \\
    \midrule
    Eckig, zweiseitig & 104 & 0.58 & 167.1 & 2361.8 \\
    Eckig, einseitig  & 100 & 0.60 & 502.4 & 1191.5 \\
    Rund, einseitig   & 100 & 0.55 & 121.3 &  541.8 \\
    \bottomrule
  \end{tabular}
\end{table}

\subsection{Bestimmung des Elastizitätsmoduls}

Von den jeweils gemessenen Werten für die Durchbiegung des Stabes werden die Werte der jeweiligen Nullmessung subtrahiert, somit werden die Werte für $D(x)$ nach

\begin{equation}
  D(x) = D_m(x) - D_\text{Null}(x)
\end{equation}

berechnet. Die Werte für $x$ werden in verschiedene Funktionen überführt, um so eine lineare Ausgleichsrechnung durchführen zu können.
Dabei ist die Funktion für den eckigen Stab mit einseitiger Einspannung

\begin{equation}
  g_\text{se}(x) = l_\text{se} \cdot x^2 - \frac{x^3}{3}
\end{equation}

und für den runden Stab mit ebenfalls einseitger Einspannung

\begin{equation}
  g_\text{sr}(x) = l_\text{sr} \cdot x^2 - \frac{x^3}{3} .
\end{equation}

Für den eckigen Stab mit beidseitiger Einspannung wird für die Seite links von dem Gewicht sowie die Seite rechts von dem Gewicht je eine Funktion benötigt.
Die Funktion für die linke Seite, also $\frac{l_d}{2} \leq x \leq l_d$, lautet

\begin{equation}
  g_\text{dl}(x) = 4x^3 - 12l_d \cdot x^2 + 9l_d^2 \cdot x - l_d^3
\end{equation}

und der entsprechende für die rechte Seite, also $0 \leq x \leq \frac{l_d}{2}$,lautet

\begin{equation}
  g_\text{dr}(x) = 3 l_d^2 \cdot x - 4x^3.
\end{equation}

Bei diesen Formeln sind $l_d$, $l_\text{sr}$ und $l_\text{se}$ jeweils die Längen der Stäbe.

\begin{table}[!htp]
\centering
\caption{Einseitig eingespannter eckiger Stab.}
\label{tab:stab2_single}
\begin{tabular}{S[table-format=2.0] S[table-format=2.2] S[table-format=1.2]}
\toprule
{$x$ / cm} & {$D_\text{Null}$ / mm} & {$D_m$ / mm} \\
\midrule
 3 &  0.00 & 0.05 \\
 6 & -0.02 & 0.17 \\
 9 & -0.03 & 0.34 \\
12 & -0.01 & 0.63 \\
14 &  0.01 & 0.86 \\
16 &  0.03 & 1.11 \\
18 &  0.08 & 1.41 \\
20 &  0.12 & 1.71 \\
22 &  0.18 & 2.06 \\
24 &  0.19 & 2.36 \\
26 &  0.19 & 2.69 \\
28 &  0.20 & 3.04 \\
30 &  0.24 & 3.42 \\
32 &  0.25 & 3.83 \\
34 &  0.25 & 4.24 \\
36 &  0.30 & 4.60 \\
38 &  0.31 & 5.00 \\
40 &  0.33 & 5.41 \\
42 &  0.32 & 5.84 \\
44 &  0.32 & 6.24 \\
46 &  0.32 & 6.64 \\
48 &  0.32 & 7.05 \\
\bottomrule
\end{tabular}
\end{table}

\begin{table}[!htp]
\centering
\caption{Einseitig eingespannter runder Stab.}
\label{tab:stab3_single}
\begin{tabular}{S[table-format=2.0] S[table-format=2.2] S[table-format=1.2]}
\toprule
{$x$ / cm} & {$D_\text{Null}$ / mm} & {$D_m$ / mm} \\
\midrule
 3 &  0.00 & 0.09 \\
 6 & -0.05 & 0.14 \\
 9 & -0.07 & 0.21 \\
12 & -0.06 & 0.51 \\
14 & -0.01 & 0.75 \\
16 &  0.05 & 1.00 \\
18 &  0.13 & 1.28 \\
20 &  0.22 & 1.60 \\
22 &  0.32 & 1.95 \\
24 &  0.42 & 2.31 \\
26 &  0.52 & 2.70 \\
28 &  0.65 & 3.10 \\
30 &  0.75 & 3.53 \\
32 &  0.87 & 3.96 \\
34 &  0.99 & 4.39 \\
36 &  1.19 & 4.86 \\
38 &  1.24 & 5.33 \\
40 &  1.34 & 5.73 \\
42 &  1.50 & 6.20 \\
44 &  1.70 & 6.75 \\
46 &  1.85 & 7.15 \\
\bottomrule
\end{tabular}
\end{table}

\begin{table}[!ht]
\centering
\caption{Zweiseitig eingespannter eckiger Stab.}
\label{tab:stab1_dual}
\begin{tabular}{S[table-format=2.0] S[table-format=2.2] S[table-format=1.2]}
\toprule
{$x$ / cm} & {$D_\text{Null}$ / mm} & {$D_m$ / mm} \\
\midrule
 3 & -0.00 & 0.20 \\
 6 & -0.12 & 0.25 \\
 9 & -0.25 & 0.28 \\
12 & -0.34 & 0.35 \\
14 & -0.31 & 0.50 \\
16 & -0.21 & 0.67 \\
18 & -0.10 & 0.86 \\
20 & -0.01 & 1.05 \\
22 &  0.14 & 1.23 \\
24 &  0.24 & 1.38 \\
26 &  0.35 & 1.56 \\
29 & -1.01 & 0.09 \\
31 & -1.00 & 0.20 \\
33 & -0.88 & 0.23 \\
35 & -0.78 & 0.26 \\
37 & -0.63 & 0.31 \\
39 & -0.56 & 0.28 \\
41 & -0.55 & 0.31 \\
43 & -0.48 & 0.28 \\
46 & -0.34 & 0.24 \\
49 & -0.16 & 0.25 \\
52 & -0.00 & 0.19 \\
\bottomrule
\end{tabular}
\end{table}

Die Werte in \autoref{tab:stab1_dual}, \autoref{tab:stab2_single} sowie \autoref{tab:stab3_single} werden genutzt um mit den oben genannten Gleichungen eine lineare Ausgleichsrechnung durchzuführen.
Die Ergebnisse werden mittels Python 3.7.0 geplottet und mit einem Fit der Form $D(x) = a * g(x) + b$ angenähert.
Die dazugehörigen Parameter $a$ und $b$ sind mittels
\begin{equation}
\label{eqn:a}
  a = \frac {\sum_{i=1}^N (D(x)_i - \overline{D(x)}) (g(x)_i - \overline{g(x)})}{\sum_{i=1}^N (D(x)_i - \overline{D(x)})^2}
\end{equation}
und
\begin{equation}
\label{eqn:b}
  b = \overline{g(x)} - a \cdot \overline{D(x)}
\end{equation}
zu berechnen.
Die entstehenden Plots sind in \autoref{fig:stab2_single}, \autoref{fig:stab3_single} sowie \autoref{fig:stab1_1_dual} und \autoref{fig:stab1_2_dual} zu sehen.

\begin{figure}
  \centering
  \includegraphics{stab2_single.pdf}
  \caption{Plot der Messwerte von dem einseitig eingespanntem eckigen Stab.}
  \label{fig:stab1-plot}
\end{figure}

\begin{figure}
  \centering
  \includegraphics{stab3_single.pdf}
  \caption{Plot der Messwerte von dem einseitig eingespanntem runden Stab.}
  \label{fig:stab1-plot}
\end{figure}

\begin{figure}
  \centering
  \includegraphics{stab1_1_dual.pdf}
  \caption{Plot der Messwerte rechts von dem Gewicht von dem zweiseitig eingespanntem eckigen Stab.}
  \label{fig:stab1_1-plot}
\end{figure}

\begin{figure}
  \centering
  \includegraphics{stab1_2_dual.pdf}
  \caption{Plot der Messwerte links von dem Gewicht von dem zweiseitig eingespanntem eckigen Stab groß.}
  \label{fig:stab1_2-plot}
\end{figure}

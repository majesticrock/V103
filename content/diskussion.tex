\section{Diskussion}
\label{sec:Diskussion}

Bei dem einseitig eingespannten, eckigen Stab handelt es sich wahrscheinlich um Messing.
Die Dichte von Messing (CuZn37) beträgt $\rho_\text{Me} = 8410 \frac{\symup{kg}}{\symup{m}^3}$ \cite{messing}. 
Der gemessene Werte von $\rho_\text{se} = 8373,3 \frac{\symup{kg}}{\symup{m}^3}$ weicht nur sehr gering von diesem ab.
Auch das Elastizitätmodul liegt innerhalb von dem in der Literatur gefundenem Rahmen von $78$ GPa bis $123$ GPa \cite{messing}.

Die Elastizitätmodule der anderen beiden Stäbe sind sehr ähnlich mit dem von Aluminium von $E_\text{Al} = 70$ GPa \cite{ela}.
Sie weichen lediglich um $-6$ \% im Falle des runden Stabes beziehungsweise um $+11$ \% im Falle des eckigen Stabes ab.
Die Dichte von Aluminium beträgt $\rho_\text{Al} = 2700 \frac{\symup{kg}}{\symup{m}^3}$ \cite{alu}.
Hier haben die Messwerte eine große Abweichung von $+43$ \% für den runden Stab und $-30$ \% für den eckigen. Da allerdings die Elastizitätmodule sehr nah an den Literaturwerten liegen, kann in diesem Fall von Messfehlern ausgegangen werden.

Fehler, die bei der Messung der Dicken der Stäbe gemacht werden, wirken sich sehr stark auf das Endergebnis aus, da die Formeln von der vierten Potenz dieser abhängen.

Eine der beiden Messuhren hängt sehr wacklig an der Apperatur und bereits Verschiebungen entlang des Stabes können die Messwerte abweichen lassen.
Stöße an den Tischen verändern den von der Uhr angezeigten Wert teilweise sehr stark.
Der Fehler, der durch die Krümmung des Stabes existiert, wird durch die Nullmessung gut herausgerechnet.

%78 - 123 GPa - Messing
%70 GPa - Aluminium

%8410 kg/m^3 - Messing
%2698 kg/m^3 - Alu